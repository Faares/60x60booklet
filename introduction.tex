\part{Introduction}
\section{About Mathematics}
Mathematics, is the art of abstracted beauty. That art which everybody in this world can understand it in a different way and they struggling in explain it to other people. When we talk mathematically, we are trying to express our self's, reaching others, and touching souls. Meanwhile, we deeply care about truth and being formal after all of that.

Learning Mathematics is the learning of how to manage our ideas and beliefs about the world, how to understands things and put them into systems gives us the ability to predict the consequences of each step we can do. Our world is rich, rich of beauty which makes us wondering how it consist, how every part interact and affect the whole. From here, Mathematics can be helpful to describe that beauty and make it measurable, usable, with accuracy.

\section{About Proof}
I mentioned that when we talk mathematically we are trying to express our self's, reaching others and we struggling in doing that. In the early history the humanity developed languages to jump over this problem and going forward, and we succeeded! But, later we faced {\it the problem of correctness and trust: How can we know if what we say correct or not? And how can we know Alice is not lying to us?} So, we needed a tool to investigate and make sure everything is as what we expected, that tool was what we call now {\it Proof}.

In every proof you will see in your life, it will starts with a common knowledge between two sides - I will call them {\it Prover} and {\it Learner} - and they already agreed on its correctness, no doubt's or arguing about that. After, the {\it Prover} begin to use that knowledge to derive a new information wasn't obvious to the {\it Learner}, by logical steps none of them can be false. As a result, the {\it Learner} should be convinced and have no doubt's about the new knowledge. Necessary to say, the last step comes with soul depth and believe.

Generally speaking, there are two way to proof something: Deduction or Induction. The first takes top-down approach while the second takes the opposite. Each one of them has several and various techniques, and its depends on the subject field and area to determine its usefulness. However, we will not go that far here and it's enough for the reader to know this. For more check {\it How to solve it} by George Pólya \cite{polyasolve}.

\section{Mathematics and Proof}
What if we mix beauty with truth? the result will be {\bf Mathematics}. Grasp the whole world, and let it down in proofed system, and enjoy with the diamond. Euclid in his Elements\cite{euclid1956elements} was the first known person who's done that, since then mathematics dressed her new form.

Mathematicians had basic units and wanted to build a skyscraper. The {\it Proof} was -and still- what deny it from falling down. Since that, the story begins.

\section{How Mathematician's really work?}\label{intro.work}
Mathematics is an art. The art of discovery, invent, abstract, generalize ideas, and finally solve problems. What really mathematician do is playing with their flight thoughts to answer a question. Mathematician's are like artist  differ in their working style, depending on their personalities, approaches, cognitive functions. Some people prefer to work on a specific problem and solve it, other people enjoy developing theories and getting things formal in a general framework, while others stays in between: anatomizing the theories for the solver's so they can solve harder problems. A mathematician becomes professional, when he started to create his tools and develop his theorems and proofs, and when you read his paper you know well his power and abilities.

They started with a well definitions, building a floor over floor, to hit a goal. Seeking for clarity or to develop a tool to be used in solving a problem: Basically they {\it Implement a Strategy}. Imagination and intuition, are the energy of Mathematician's, learning is the foot, writing is the hand, and finally: a proof or solution is the product.

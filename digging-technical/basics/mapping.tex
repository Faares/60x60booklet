
\subsection{Mapping}
I assume by now, you noticed that we started by a collection of things - the set elements -, putted them into a bag - the set -, and lastly associate its elements with each other - by relationship -. All of that's gaves us a vagus pattern we can determine it by the relations properties we learned! 
Eventually all of what we have done so far are an internal study, we didn't do any external movements on the elements, like: transfer, shifting, or even combined two elements to produce one. Mapping and Operation are the two mathematical concepts did that.

\subsubsection{What is mapping?}
I like the term {\it Mapping}, because its so fruitful semantically. When we lost in the desert, we can use a {\it Map} to decide which direction have to take, the same map takes a piece of Earth surface and {\it associate} each point of it with a point on a piece of paper. That's mapping!

Mapping is all about {\it doing association on a certain condition or rule}. Something like: Take a point from the Earth surface, scale it down by a specific ratio, then draw it again on the same position in the paper.

Some people call it {\it Function} other {\it Mapping}, but still it's the same concept. Formally, you must have two sets -not necessarily different- the {\it Origin} set and {\it Destination} set, with a formula or algorithm to map the elements from the {\it Origin} to the {\it Destination}. Mathematician differs in naming them, but the concepts are the same again.

\paragraph{Mapping.} 
Given two sets $O$,$D$ we say $f(x)$ is a map from $O$ to $D$ if: 
    for every element $o \in O$ there exist a destination element $d \in D$, such that $f(o)=d$.

\subsubsection{Types of Mapping}
There are several types of mapping \footnote{Actually, mathematician's admiring define new types and create their own tools, see section \ref{intro.work}.}, here are the important one's:
\begin{itemize}
    \item A map $f(x)$ called {\bf Injective} if every \underline{associated} element in the {\it Destination} set associated with \underline{only one element} in the {\it Original} set. Formally, if $o_1 = o_2$ then $f(o_1) = f(o_2)$.
    \item A map $f(x)$ called {\bf Surjective} if every element in the {\it Destination} set associated with an element in the {\it Original} set. Formally, for every $d \in D$ there exist $o \in O$ such that $f(o)=d$ or $f(O)=D$.
    \item A map $f(x)$ called {\bf Bijective} if it injective \underline{and} surjective.\footnote{Other books maybe call the three as {\it  one to one, onto, and one to one and onto} respectively.}
\end{itemize}
An interesting to say, there are bunch of branches in mathematics specialized in Mapping and their types. It's a fundamental concept in modern mathematics, almost any branch in mathematics have his own functions, their types and applications.
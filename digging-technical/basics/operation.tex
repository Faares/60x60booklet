\subsection{Operation}
We saw that mapping is the way to {\it do something}, takes one -or more than one from the same set- input  and produce an output. {\it Operation} is similar to map, but with no restriction on the original set. Generally, an {\it Operation} could takes different inputs from different sets, and produce one output.

\subsubsection{Abstract and Generalize}
A good example for operation is the basic arithmetic operations {\it Addition} and {\it Subtraction}, you can add or subtract numbers from different sets, for example define addition operation as $+_1:\mathbb{Z}\times \mathbb{Q}\longrightarrow\mathbb{Q}$ or $+_2:V\footnote{$V$ refers to Vector Space. } \times V\longrightarrow V$, they do the same thing, but on different sets which eventually leads to distinct results. In {\it Operation} we generalize our naive concept about addition and subtraction..etc to new area: It's not bounded by anything except the nature of sets elements: numbers, sets, vectors..etc. In other word, we abstract the nature of the operation and apply it to different objects.

To state the obvious, $+_1$ will takes two numbers, one from integers the other from rational, and produce of course a rational number. But $+_2$ even it's an addition, but on vectors instead of numbers, taking two vectors and return a new vector.

One can think of something like: {\it Can we takes two elements from the same set, and produce a completely different object?}. Actually, that's one of the powerful properties operation gave us. Take for example what we call in Vector Calculus the {\it Scalar Product} defined as $\mathbb{\cdot}:V\times V\longrightarrow\mathbb{R}$ which takes two vectors and return a scalar quantity can be used to answer: does they perpendicular or not?\footnote{A famous result in Vector Calculus, states that two vectors $a,b \in \mathbb{V}$ are perpendicular if their scalar product equal zero.}

\subsubsection{Operation as Algebraic System}
When we equip a set $S$ with an operation $\cdot$ like this $(S,\cdot)$ we create what we call a {\it System}. A number of elements with specific operation let us manipulate them under.

There is an interesting property if it occurs in the system, that leads to be mathematically impressive. They call it {\it Closure property}, and we say {\it the system is algebraically closed} or {\it the system is closed} if the operation defined takes elements from the equipped set and output an element in the same set, symbolically $\cdot: S\times S\longrightarrow S$\footnote{Nice to mention that The Fundamental Theorem of Algebra states that the Complex Numbers $\mathbb{C}$ are algebraically closed, see \cite{enwiki:1023272575}.}.

Whenever the system satisfying the closure property, nothing strange will appears -- No solutions, results, or even proof will leads to undefined object. To illustrate this, imagine you are a mathematician in the middle ages and wanted to solve the quadratic equation: $$(1)x^2+(0)x+(1)=0 \Longrightarrow x^{2}+1=0 \Longrightarrow x^{2}=-1$$ 
at that time, Complex Numbers $\mathbb{C}$ wasn't invented\footnote{Or discovered, I don't like that byzantine discussion; Does the math invented or discovered?} yet. So, all solutions must be in $\mathbb{R}$, but there is no $x \in \mathbb{R}$ such that $x^{2}=-1$. The Closure property emphasize this will not happen if the system satisfy it.

Moreover, if the system $(S,\cdot)$ wasn't closed under that operation we cannot procedure for more than one operation. Since the output element of the firsts step  {\it is not in $S$}.\footnote{Take $+:V\times V\longrightarrow \mathbb{R}$ as example.}. 

The closure property is not the only property for operations, there are various and several properties. Its helpful when it becomes to prove something and write a strategy, like {\it Associativity, Commutativity }.
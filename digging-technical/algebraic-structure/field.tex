\section{Fields \label{fields}}

\subsection{Prologue}
The end of the last section was tricky, and it was intended for a purpose. Fields are very related to Rings, it developed over Rings, unlike the relation Rings and Groups. Every Field is a Ring with certain restriction conditions.

Mainly, Fields developed to solve polynomials equations, side by side with Galois Theory, and other theories developed to solve polynomials equations with degrees 4,5, and more. But, I would rather -and prefer- to introduce it as a geometric concept instead of an algebraic. Why? I think its properties have geometric richness over algebraic. Even if we introduced the {\it extension} idea, it's clear that "Bigger"\footnote{For example, $\mathbb{C}$ is the extension field of $\mathbb{R}$, which can be represented by $0i+x, x \in \mathbb{R}$ which is a line in the Complex Plane, the same for $\mathbb{Q}$ with respect to $\mathbb{R}$, it's a line with "gap points -- irrationals", $\mathbb{R}$ is the extension field of $\mathbb{Q}$.  } is what you have to pass through, needless to mention the relationship between the degree of the extension field and Vector Space.

Fields, as far as I can see, is the new form of the connected line between Numbers and Geometry, Discrete and Continues, a new and more abstracted form of $xy$-plane. Usually, we use $\mathbb{R}$ with simple $xy$-plane, now we use Field $\mathbb{F}$ with a set of vectors $V$ to define a Vector Space $\mathbb{V}= (V, \mathbb{F})$ -- The generalization of $xy$-plane.

\subsection{Fields, Formally}
We already defined Fields using Rings. However, in mathematics we can always do the same job with a different approach. Here, we will define Fields by axioms instead of Rings with restricted conditions\footnote{Ring, satisfying Integral Domain and Division Ring conditions.}.

Lets say we have a system $(F,+,\cdot)$, we call it {\it Field} if and only if:
\begin{itemize}
    \item Axioms for $(F,+)$:
    \begin{enumerate}
        \item $(F,+)$ form Abelian Group.
    \end{enumerate}
    \item Axioms for $(F,\cdot)$:
    \begin{enumerate}
        \item $\cdot$ is Associative.
        \item $\cdot$ is Commutative.
        \item There exist $0 \neq 1 \in F$ such that $1a = a1 = a$ for any $a \in F$.
        \item There exist $a^{-1} \in F$ such that $a^{-1}a = aa^{-1} = 1$ for any $0 \neq a \in F$.
    \end{enumerate}
    \item Axioms for the relation between $+$ and $\cdot$:
    \begin{enumerate}
        \item $1 \neq 0$ -- The addition and multiplication identities are distinct.
        \item The multiplication is distributed over addition.
    \end{enumerate}
\end{itemize}
That's it! It's a system, where you can add, subtract, multiply, and divide.
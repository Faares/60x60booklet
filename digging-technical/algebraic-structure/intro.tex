\part{Algebraic Structures}
\section{Introduction}
We discussed the closure property, and saw what does it emphasized. Backing to the history, algebra especially, and lets start thinking again of equations in more general way. Like what if we use the concept of {\it Operations} instead of the elementary and naive concepts? I mean, what if we deal with {\it Addition} as an {\it Original Operation} and {\it Subtraction} as {\it Inverse Operation}? by this, we clear the vague pattern into what we call {\it Algebraic Structure}.

\subsection{What is The Algebraic Structure?}
The {\it Algebraic Structure} is an algebraic system with more generalization for the operation, and restriction conditions to hold that structure. Maybe we have an algebraic system, with algebraic structure maybe not. Moreover, the system could have only a subset holds that structure, and maybe it contains more than one structure.

In simple words: its a clear pattern frequently appears, can be defined formally using axioms, and digging into by Theorems and Lemmas. Symbolically:

\paragraph{Algebraic Structure.} Let $(S,\cdot)$ be an algebraic system, and $\set{P_1(x),P_2(x),...}$
be the set of axioms define a structure. We say $(S,\cdot)$ is an {\it Algebraic Structure or Structure} if and only if the system satisfy the set of axioms -- i.e None of them have $False$ as truth value. 

\subsection{The Benefit of Algebraic Structure}
The main idea behind structures is to develop a deep understanding of a general system rather than a special one. Every structure has a huge Theory contains its proved theorems and properties, so if the algebraic system satisfies the structure definition, we directly apply it and know the system properties. Only by checking the definition!

After learning one or two structures, it will become clear why we care so much about, and how it can expand our research to new areas sometimes we couldn't expect it. Let's dive into it!
\section{Rings \label{ring}}
\subsection{Prologue}
Until now, we worked on a system with one operation. Rings, unlike Group it has two equipped operations. Essentially, it developed to generalize two main ideas: number of operations, and Addition with Multiplication idea, especially the second, its the abstraction of them. As in \cite{61505} they name it "Ring" because of that closure property, with circulation. Rings is important in studying Polynomials\footnote{Recall Polynomial formula: $a_nx^n+a_{n-1}x^{n-1}+...+a_0x^{0}$}, Algebraic Geometry, and a variant of subjects.

Since Rings as mentioned has two operation, we have to be careful about their relationship. We will see how it should be.
\subsection{Rings, Formally}
The axioms of Rings can be divided into three parts: axioms for Addition, Multiplications, and their relationship.

We say $(R,+,\cdot)$ is a Ring, if it satisfy:
\begin{itemize}
    \item $+$ axioms:
    \begin{itemize}
        \item $(R,+)$ is Group. 
        \item And for all $a,b \in R$: $a+b=b+a$\footnote{If any Group satisfy this property, we call it {\it Commutative Group} or {\it Abelian Group}.}.
    \end{itemize}
    \item $\cdot$ axioms:
    \begin{itemize}
        \item For any $a,b,c \in R$: $a \cdot (b \cdot c) = (a \cdot b) \cdot c$ -- Associativity of $\cdot$, the order doesn't matter.
    \end{itemize}
    
    \item $+,\cdot $ relationship axioms:
    \begin{itemize}
        \item For any $a,b,c \in R$: $a \cdot (b + c) = a \cdot b + a \cdot c$ -- Right Distributivity.
        \item For any $a,b,c \in R$: $(a + b) \cdot c = a \cdot c + b \cdot c$ -- Left Distributivity.
    \end{itemize}
\end{itemize}

In simple words: A structure, you can add, subtract, multiply in it, and the multiplication can be distributed on the addition. A good example for it is $(\mathbb{Z},+,\cdot)$ with the arithmetic addition and multiplication, you can try and prove it.

I can see Rings as an abstraction of Real Line, you can go far away either left or right, slowly by addition, quickly by multiplication, hurry by distributivity,  and reverse the direction using subtraction. But wait, why we don't see division? the answer, is because of "Gaps" caused by Undefined Division, like $\frac{1}{0}$. Fortunately, There is a special type of Rings, called {\it Division Ring} which the division operation is possible, since for every $a \in R$, there exist multiplicative inverse $a^{-1} \neq 0$ such that $aa^{-1}=a^{-1}a=1$ where $1$ is the multiplicative identity.

Using this formalization, we can derive special types or Rings like {\it Divison Ring}, lets illustrate it: \begin{enumerate}
    \item First we have {\it Rings}.
    \begin{enumerate}
        \item If it has a Multiplicative Identity not equal to zero, we call it {\it Ring with Identity}.
            \begin{enumerate}
                \item If it Ring with Identity, and every nonzero element has a multiplicative inverse, we call it {\it Division Ring}. 
            \end{enumerate}
        \item If for every two element in the Ring, $ab=ba$, we call it {\it Commutative Ring}.
        \item If it (a.i)-- Ring with Identity and (b)-- Commutative Ring, we call it {\it Integral Domain}\footnote{To understands the meaning behind {\it Integral Domain}, I refer the reader to this glamorous explanation \cite{46944}.}.
    \end{enumerate}
    \item Lastly, we have {\it Fields}, the next structure we will introduce. A Ring is Field, if and only if it's {\it Integral Domain} and {\it Division Ring}.
\end{enumerate}

This derivation could be ambiguous right now, but we will light it up later. Looking clearly to the beauty of mathematics.
\documentclass{article}
\usepackage[utf8]{inputenc}
\usepackage{biblatex}
\usepackage{braket}
\usepackage{titlesec}
\usepackage{hyperref}
\usepackage{amsmath}
\usepackage{amssymb}
\addbibresource{ref.bib}

% Footnote symbol.
\makeatletter
\def\@fnsymbol#1{\ensuremath{\ifcase#1\or \dagger\or \ddagger\or
   \mathsection\or \mathparagraph\or \|\or **\or \dagger\dagger
   \or \ddagger\ddagger \else\@ctrerr\fi}}
\makeatother
    
\title{A light introduction to Advance Mathematics: an Abstract Algebra Approach}

\author{Fares AlHarbi \thanks{Independent, \href{mailto:faares@acm.org}{Faares@acm.org}.}}
\date{August 2021}

\begin{document}

\maketitle
\begin{abstract}
    This booklet written to be used in \hyperlink{https://salla.sa/durba/RYePmz}{60x60} event,
    to provide the big picture of advance mathematics and how mathematician's really work.
    No deep technical nor tough subjects, It's written with intuition and simplicity in mind. 
\end{abstract}

\clearpage
\tableofcontents 

% Booklet introduction
\part{Introduction}
\section{About Mathematics}
Mathematics, is the art of abstracted beauty. That art which everybody in this world can understand it in a different way and they struggling in explain it to other people. When we talk mathematically, we are trying to express our self's, reaching others, and touching souls. Meanwhile, we deeply care about truth and being formal after all of that.

Learning Mathematics is the learning of how to manage our ideas and beliefs about the world, how to understands things and put them into systems gives us the ability to predict the consequences of each step we can do. Our world is rich, rich of beauty which makes us wondering how it consist, how every part interact with and affect the whole. From here, Mathematics can be helpful to describe that beauty and make it measurable, usable, with accuracy.

\section{About Proof}
I mentioned that when we talk mathematically we are trying to express our self's, reaching others and we struggling in doing that. In the early history the humanity developed languages to jump over this problem and going forward, and we succeeded! But, later we faced {\it the problem of correctness and trust: How can we know if what we say correct or not? And how can we know Alice is not lying to us?} So, we needed a tool to investigate and make sure everything is as what we expected, that tool was what we call now {\it Proof}.

In every proof you will see in your life, it will starts with a common knowledge between two sides - I will call them {\it Prover} and {\it Learner} - and they already agreed on its correctness, no doubt's or arguing about that. After, the {\it Prover} begin to use that knowledge to derive a new information wasn't obvious to the {\it Learner}, by logical steps none of them can be false. As a result, the {\it Learner} should be convinced and have no doubt's about the new knowledge. Necessary to say, the last step comes with soul depth and believe.

Generally speaking, there are two way to proof something: Deduction or Induction. The first takes top-down approach while the second takes the opposite. Each one of them has several and various techniques, and its depends on the subject field and area to determine its usefulness. However, we will not go that far here and it's enough for the reader to know this. For more check {\it How to solve it} by George Pólya \cite{polyasolve}.

\section{Mathematics and Proof}
What if we mix beauty with truth? the result will be {\bf Mathematics}. Grasp the whole world, and let it down in proofed system, and enjoy with the diamond. Euclid in his Elements\cite{euclid1956elements} was the first known person who's done that, since then mathematics dressed her new form.

Mathematicians had basic units and wanted to build a skyscraper. The {\it Proof} was -and still- what deny it from falling down. Since that, the story begins.

\section{How Mathematician's really work?}\label{intro.work}
Mathematics is an art. The art of discovery, invent, abstract, generalize ideas, and finally solve problems. What really mathematicians do is playing with their flight thoughts to answer a question. Like artists, Mathematicians are  differ in their working style, depending on their personalities, approaches, cognitive functions. Some people prefer to work on a specific problem and solve it, other people enjoy developing theories and getting things formal in a general framework, while others stays in between: anatomizing the theories for the solver's so they can solve harder problems. A mathematician becomes professional when he started to create his tools and develop his theorems and proofs, You can figure out his power and abilities when you reading his papers.

They -Mathematician- started with a well definitions, building a floor over floor, to hit a goal. Seeking for clarity or to develop a tool to be used in solving a problem: Basically they {\it Implement a Strategy}. Imagination and intuition, are the energy of Mathematician's, learning is the foot, writing is the hand, and finally: a proof or solution is the product.


% Technical introduction
\part{Digging Technical}

\section{Understanding Mathematics}
Mathematician built a skyscraper over centuries, and in order to reach the wanted floor you have to pass through at least the requisite floors first. The learning process in mathematics is hierarchical, whenever you are well and good in the basics, you will get O-Ah! moment faster. In order to treat complicated things, we need to have tools process it accurately like surgical instrument in surgery. 

Over the history, Mathematics developed from vocabulary in Al-Khawarizmi era,  tell quantitize in Scientific Revolution era in 16-17th century, and finally symbolism in 18-19th century by Cantor, Whitehead, Russell and others - From them, we begin.

Before going towards, I assume the reader is already grasp the essential rule of Logic in Mathematics, since I will not explain it here because in my opinion Logic is intuitive and people don't need to learn it.
\section{Sets, Relations, Mapping}
Mathematics developed so fast, and grow in way opened the doors for more generalization and invent or discover different treatments. {\it Set Theory} was the first room to enter.

\subsection{Sets}
What is {\it set}? Generally speaking, its a collection of things. I would rather say a {\it Container} for objects, we put things in and work on it. Symbolically:
$$
S = \set{x \mid P(x)}
$$
where $P(x)$ the characterization property i.e. what decide does the element can be in the set or not.
\subsubsection{Basic Concepts}
\paragraph{Membership.}
Let's say we have an object $y$ , if the object belongs to the set -i.e. satisfying the $P(x)$- we write $y \in S$ if not $y \not\in S$.
\paragraph{Union.}
If we have $A,B,C$ as sets, and we wanted to group all of their elements in one set $S$. How to do that? We use the {\it Union} operation, symbolically:
\begin{align*}
        \space A = \set{a} , && B = \set{b} , && C = \set{c} \\
        \\ 
        A \cup B = \set{a,b} && A \cup C = \set{a,c} && B \cup C = \set{b,c} \\
        \\
        && A \cup B \cup C = \set{a,b,c}
\end{align*}
\paragraph{Intersection.}
Same as {\it Union } except we want to group all of the common elements between sets:
\begin{align*}
        \space A = \set{a,b} , && B = \set{b,c} , && C = \set{c} \\
        \\ 
        A \cap B = \set{b} && A \cap C = \set{ } = \phi && B \cap C = \set{c} \\
        \\
        && A \cap B \cap C = \set{ } = \phi
\end{align*}
\paragraph{Subset.}
TODO.
\subsubsection{Special Sets}
There are special sets in Mathematics, we talk about and use them on a daily basis. Here are some:
\begin{center}
\begin{tabular}{| c | c | c | }
\hline
Symbol & Name & Definition \\
\hline
$\mathbb{N}$      & Natural Numbers & $\set{1,2,3,4,5,6,..}$ \\
$\mathbb{Z}$      & Integers  & $\set{..,-2,-1,0,1,2,..}$        \\
$\mathbb{Q}$      & Rational Numbers & $\set{ \frac{a}{b} \mid a,b \in \mathbb{Z}, b \neq 0}$  \\
$\mathbb{I}$      & Irrational Numbers & $\set{ x \mid x \not\in \mathbb{Q}}$ \\
$\mathbb{R}$      & Real Numbers  &  $\mathbb{Q} \cup \mathbb{I}$  \\
$\mathbb{C}$      & Complex Numbers & $\set{a+ib \mid a,b \in \mathbb{R}, i=\sqrt{-1}}$    \\
$\mathbb{\phi}$   & Empty set   &   The nothing set! e.g. $A \cap \  C$ \\
\hline
\end{tabular}
\end{center}
Its enough for the reader to know this, Set Theory is a huge topic I can't cover it all here.

\clearpage
\subsection{Relations}

\subsubsection{Ordered Pair}
We already defined sets and learned the basic concepts. But until now we didn't study how they interact with each other and what's the nature of it. To accomplish that, we need to introduce a new concept associate an element from set $A$ with an element from $B$ set or from and into the set of it self e.g. from $A$ to $A$.
This new concept called {\it Ordered Pair}, symbolically:
$$
    (a,b) \mid a \in A, b \in B
$$
Whenever we write it, we mean there exist an association between $a$ and $b$. A marked note to remember: {\it The order of the element is important:
$$
    (a,b) \neq (b,a)
$$
} The reason behind is because the right hand side (RHS) associate $a$ with $b$, while the left hand side (LHS) associate $b$ with $a$ and the meaning is broadly different! as we will see later \footnote{in \ref{relation.define} Examples, you can see it clearly with order relationships.}. Needless to say, we didn't specify the nature of that association yet.
\subsubsection{Define Relationship \label{relation.define}}
The {\it Ordered Pair} associate one element from a set with a one from another. What about sets? how can we associate a set with another? the answer is by {\it Cartesian Product}: the set of all possible ordered pairs combined from the two sets. Symbolically:
$$
A \times B = \set{(a,b) \mid a \in A \ and \  b \in B}
$$

A relationship $R$ is {\it the set of ordered pairs $(a,b)$ satisfying $aRb$, which is a subset of $A \times B$}. 
\paragraph{Examples. }
\begin{align*}
        \space A = \set{1,2} , && B = \set{2,3} , && C = \set{4,5}
\end{align*}
\begin{align*}
        A \times B = \set{(1,2),(1,3),(2,2),(2,3)} \\
        B \times A = \set{(2,1),(3,1),(2,2),(3,2)} \\
        % A \times A = \set{(1,1),(1,2),(2,1),(2,2)} \\
\end{align*}
\begin{itemize}
    \item Equality relationship ($=$):
        \begin{itemize}
            \item  on $A \times B$: $\set{(2,2)}$.
            \item on $B \times A$: $\set{(2,2)}$.
        \end{itemize}
    \item Less than relationship ($<$):
        \begin{itemize}
            \item  on $A \times B$: $\set{(1,2),(1,3),(2,3)}$.
            \item  on $B \times A$: $\phi$.
        \end{itemize}
    \item Less than or equal relation ($>$):
        \begin{itemize}
            \item  on $A \times B$: $\phi$.
            \item  on $B \times A$: $\set{(2,1),(3,1),(3,2)}$.
        \end{itemize}
    \item Less than or equal relation ($\leq$):
        \begin{itemize}
            \item  on $A \times B$: $\set{(1,2),(1,3),(2,2),(2,3)}$.
            \item  on $B \times A$: $\set{(2,2)}$.
        \end{itemize}
\end{itemize}

\subsubsection{Properties of Relationships}
% We defined a relationship, now it's the time to study it! mathematician's founds a general properties help us in determining the behaviour of the relationship, moreover, it's help us to define a complex relationship gives us a wonderful results as we will see later \footnote{In the next section, we will see how the equivalence relation gives us a partition on the set.}.

When we define a relation on a set, we create a vagus patterns. Figuring it out help us to know {\it how the elements of the set are behaves under that relationship}, which gives a rich and clear understanding of the set objects. Moreover, it will expands our knowledge and maybe knock closed doors in research.

For technical reasons, we will assume there exist a set $S$ for the following definition.
\paragraph{Reflexivity} 
A relation $R$ is {\bf Reflexive} relation if and only if for every element in the set $S$ is associated with it self. Formally $aRa$ for all $a \in S$.
\paragraph{Symmetry} 
A relation $R$ is {\bf Symmetric} relation if and only if $a$ associated with $b$ then $b$ is also associated with $a$, Formally if $aRb$ then $bRa$.

\paragraph{Transitivity} 
A relation $R$ is {\bf Transitive} relation if and only if $a$ associated with $b$ and $b$ associated with $c$, then $a$ also associated with $c$. Formally if $aRb$ and $bRc$ then $aRc$

\paragraph{Order Relations} Some relations rearrange the elements to a specific order. For example, take less than or equal $\leq$ relation on $\mathbb{Z}$, you can easily figure out that for any $z_1,z_2 \in Z$ either $z_1 \leq z_2$ or $z_2 \leq z_1$, these relations are  very useful when working on something like Lattice Theory \cite{enwiki:945896905} and others \cite{enwiki:1038741711}! We will not going forward, it's enough to know these relations for now.
\vspace{20pt}





% \subsubsection{Equivalence and Partition}
 
\clearpage

\subsection{Mapping}
I assume by now, you noticed that we started by a collection of things - the set elements -, putted them into a bag - the set -, and lastly associate its elements with each other - by relationship -. All of that's gaves us a vagus pattern we can determine it by the relations properties we learned! 
Eventually all of what we have done so far are an internal study, we didn't do any external movements on the elements, like: transfer, shifting, or even combined two elements to produce one. Mapping and Operation are the two mathematical concepts did that.

\subsubsection{Mapping, What is that?}
I like the term {\it Mapping}, because its so fruitful semantically. When we lost in the desert, we can use a {\it Map} to decide which direction have to take, the same map takes a piece of Earth surface and {\it associate} each point of it with a point on a piece of paper. That's mapping!

Mapping is all about {\it doing association on a certain condition or rule}. Something like: Take a point from the Earth surface, scale it down by a specific ratio, then draw it again on the same position in the paper.

Some people call it {\it Function} other {\it Mapping}, but still it's the same concept. Formally, you must have two sets -not necessarily different- the {\it Origin} set and {\it Destination} set, with a formula or algorithm to map the elements from the {\it Origin} to the {\it Destination}. Mathematician differs in naming them, but the concepts are the same again.

\paragraph{Mapping.} 
Given two sets $O$,$D$ we say $f(x)$ is a map from $O$ to $D$ if: 
    
    \underline{for every element $o \in O$} there exist \underline{a one and only one} destination element $d \in D$, such that $f(o)=d$.

\subsubsection{Types of Mapping}
There are several types of mapping \footnote{Actually, mathematicians admiring define new types and create their own tools, see section \ref{intro.work}.}, here are the important one's:
\begin{itemize}
    \item A map $f(x)$ called {\bf Injective} if every \underline{associated} element in the {\it Destination} set associated with \underline{only one element} in the {\it Original} set. Formally, if $o_1 = o_2$ then $f(o_1) = f(o_2)$.
    \item A map $f(x)$ called {\bf Surjective} if every element in the {\it Destination} set associated with an element in the {\it Original} set. Formally, for every $d \in D$ there exist $o \in O$ such that $f(o)=d$ or $f(O)=D$.
    \item A map $f(x)$ called {\bf Bijective} if it both injective \underline{and} surjective.\footnote{Other books maybe call the three as {\it  one to one, onto, and one to one and onto} respectively.}
\end{itemize}
\subsubsection{The Inverse Map} You can think and ask your self: Ok, I mapped an element $o \in$ {\it Original} to another $f(o) = d \in$ {\it Destination}, how can I get $o$ If I know $d = f(o)$? 

You have $d$, and $f$ only. To answer this question, the first thing you have to do is to check the map $f$ it self. To decide does it {\bf Injective} or not. Why? assume $f$ not injective, that means $d$ could be associated with two different element in the {\it Original} set, so you cannot determine which one is the correct one! Another reason, is because if we associate two elements with $d$ that deny us from defining a mapping from the {\it Destination } set to the {\it Original} set, since it violate the \underline{the only and only one} rule.

Secondly, you have to check does the map $f$ {\bf Surjective} or not. Why? Again you have $d \in$ {\it Destination} and $f$ only. If $f$ not surjective, that means $d$ may not associated with any element at all! You will waste your time and get nothing, because nothing was there at all! Another reason, is because that also will deny us from defining a map from the {\it Destination } set to the {\it Original} set, since it violate the \underline{every element in {\it Original} must associate with another} rule.

If the map $f$ passed the previous two tests, that means it's {\bf Bijective} and now you can define what we call the {\bf Inverse Map} denoted by $f^{-1}$. The main goal of this map is to reverse the original map operation. For example let say we have a map $f(x)=2x$ from $\mathbb{Z}$ to $\mathbb{Z}$, the inverse map will be $f^{-1}(x)=\frac{x}{2}$, in other word if we composite the two maps we will get the base element: $f(1)=2(1)=2$ and $f^{-1}(2)=\frac{2}{2}=1$, so $f^{-1}(f(1))=\frac{2(1)}{2}=\frac{2}{2}=1$.

Clearly not all maps have an inverse map. But of course, if the map bijective then it have one. The idea behind finding the inverse map if it passed the tests above is very basic, {\it by reversing all the operations in the original map with the reverse order}. For example if you see addition reverse it by subtraction, multiplication by division, multiplication then addition by subtraction then division \footnote{The reverse concept is very importation especially in Algebra and any algebraic system overall. We will study it widely later, so make sure you understands this section well.}.

An interesting to say, there are bunch of branches in mathematics specialized in Mapping and their types. It's a fundamental concept in modern mathematics, almost any branch in mathematics have his own functions, their types and applications.
\clearpage
\subsection{Operation}
We saw that mapping is the way to {\it do something}, takes one -or more than one from the same set- input  and produce an output. {\it Operation} is similar to map, but with no restriction on the original set. Generally, an {\it Operation} could takes different inputs from different sets, and produce one output.

\subsubsection{Abstract and Generalize}
A good example for operation is the basic arithmetic operations {\it Addition} and {\it Subtraction}, you can add or subtract numbers from different sets, for example define addition operation as $+_1:\mathbb{Z}\times \mathbb{Q}\longrightarrow\mathbb{Q}$ or $+_2:V\footnote{$V$ refers to Vector Space. } \times V\longrightarrow V$, they do the same thing, but on different sets which eventually leads to distinct results. In {\it Operation} we generalize our naive concept about addition and subtraction..etc to new area: It's not bounded by anything except the nature of sets elements: numbers, sets, vectors..etc. In other word, we abstract the nature of the operation and apply it to different objects.

To state the obvious, $+_1$ will takes two numbers, one from integers the other from rational, and produce of course a rational number. But $+_2$ even it's an addition, but on vectors instead of numbers, taking two vectors and return a new vector.

One can think of something like: {\it Can we takes two elements from the same set, and produce a completely different object?}. Actually, that's one of the powerful properties operation gave us. Take for example what we call in Vector Calculus the {\it Scalar Product} defined as $\mathbb{\cdot}:V\times V\longrightarrow\mathbb{R}$ which takes two vectors and return a scalar quantity can be used to answer: does they perpendicular or not?\footnote{A famous result in Vector Calculus, states that two vectors $a,b \in \mathbb{V}$ are perpendicular if their scalar product equal zero.}

\subsubsection{Operation as Algebraic System}
When we equip a set $S$ with an operation $\cdot$ like this $(S,\cdot)$ we create what we call a {\it System}. A number of elements with specific operation let us manipulate them under.

There is an interesting property if it occurs in the system, that leads to be mathematically impressive. They call it {\it Closure property}, and we say {\it the system is algebraically closed} or {\it the system is closed} if the operation defined takes elements from the equipped set and output an element in the same set, symbolically $\cdot: S\times S\longrightarrow S$\footnote{Nice to mention that The Fundamental Theorem of Algebra states that the Complex Numbers $\mathbb{C}$ are algebraically closed, see \cite{enwiki:1023272575}.}.
\clearpage
\section{Algebraic Structure's}

\subsection{Group}

\subsection{Rings}

\subsection{Fields}

\clearpage
\printbibliography[heading=bibintoc]

\end{document}
